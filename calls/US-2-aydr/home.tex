
\section{RENDIMIENTO ACADÉMICO}
%Máximo: 10 puntos

\subsection{Expediente académico}
%	Máximo: 4 puntos
En candidato ha cursado los siguientes estudios universitarios de pregrado y postgrado:\\

\startitems
\printStudies{superior}    
\printStudies{master}
\stopitems 


\subsection{Tesis Doctoral} 
%Máximo: 3 puntos
El candidato obtuvo los siguientes meritos durante su doctorado en cotutela:\\
 
\startitems
\printStudies{phd}
\stopitems


\subsection{Premios académicos} 	
%Máximo: 2 puntos
La tesis del candidato y su actuación como alumno de master ha sido reconocida con varios premios prestigiosos tanto oficiales como no oficiales. Los meritos relativos a la tesis se encuentran en la sección \ref{sec:premiostesis}. \\ 

\startitems 
\printStudies{prize}
\stopitems

\section{ACTIVIDAD DOCENTE}	

\subsection{Docencia en asignaturas regladas}
\startitems 
\printTeaching
\stopitems

\subsection{Hoja de Servicio}

\subsubsection{Hoja de Servicio Université de Rennes 1 e INSA}
\insertpage{1}{documentos/docencia/docencia_impartida/contratoInriaRaw.pdf}

\subsubsection{Hoja de Servicio Sevilla}
\startitems
\printOthers{hojaservicios}
\stopitems

\subsection{Evaluaciones de la docencia}
\startitems
\printTeachingQuality
\stopitems
 
\subsection{Otros Méritos}
\subsubsection{Trabajos fin de grado o master dirigidos}
\startitems
\printOthers{tfg/tfm}
\stopitems

\subsubsection{Proyectos de innovación y mejora docente}
\startitems
\printOthers{InnovacionDocente}
\stopitems 
 
\subsubsection{Elaboración material docente}
\startitems
\printOthers{Material}
\stopitems

\subsubsection{Curso de formación e innovación docente}
\startitems
\printOthers{Cursos impartidos}
\stopitems
 
\subsubsection{Otros méritos}
\startitems
\printOthers{Cursos recibidos docencia}
\stopitems
 
\section{ACTIVIDAD INVESTIGADORA}
\subsection{LIBROS} 
%. Hasta 2 puntos por cada uno.
\startitems
\printPapers{chapters} 
\stopitems
 
\subsection{ARTÍCULOS PUBLICADOS EN REVISTAS CIENTÍFICAS}


%. Hasta 5 puntos por cada uno.
%\printPapers{journalsJCR}\\
El candidato ha publicado un total de cuatro artículos en revistas del primer cuartil, cuatro en revistas del segundo y un artículo en una revista indexada en el tercer cuartil dentro del índice JCR. Asimismo, publicó un artículo en una revista no indexada. \\
\startitems
\printPapers{journals}
\stopitems 
\newpage
~
\newpage


\addtocounter{subsection}{-1}
\subsection{Conferencias muy relevantes y equiparables a revistas JCR de acuerdo al la SCIE} 
La sociedad científica de ingeniería informática ha propuesto un ranking para aquellas 
conferencias muy relevantes en el ambito de informática. Estás conferencias se equiparan a revistas científicas JCR. 
La resolución adoptada por la ANECA puede encontrarse aquí: \\
\url{http://www.scie.es/ranking-de-congresos-relevantes-para-la-scie/}.
El ranking se puede consultar aquí:\\
\url{http://gii-grin-scie-rating.scie.es/}

\startitems
\printPapers{plusconferences} 
\stopitems



\subsection{PARTICIPACIÓN EN PROYECTOS DE INVESTIGACIÓN FINANCIADOS Y EJECUTADOS}
%. Hasta 1,5 puntos por cada uno.
\label{sec:motiv}
\label{sec:cybersecurity}
\label{sec:cyberdefense}

El candidato ha participado en un total de diez proyectos de investigación en el ámbito autonómico, estatal y europeo. \\
\startitems
\printProjects 
\stopitems 

\subsection{PERTENENCIA A GRUPOS DE INVESTIGACIÓN}
El candidato ha pertenecido a los siguientes grupos de investigación de distintos países desde que inició su carrera investigadora. \\
\startitems
\printGroups 
\stopitems
 
\subsection{ESTANCIAS EN CENTROS DE INVESTIGACIÓN}
El candidato ha desarrollado gran parte de su carrera investigadora en centros internacionales de investigación. Concretamente en:\\
%. Hasta 5 puntos.
\startitems
\printStays 
\stopitems

\subsection{Ponencias}
\label{ponencias}
\insertpage{1}{./documentos/otros/DeclaracionJurada.pdf}
\newpage

\subsection{CONGRESOS Y CONFERENCIAS CIENTIFICAS}
%. Hasta 4 puntos.
A parte de las contribuciones realizadas en revistas de impacto y conferencias muy relevantes, el candidato ha publicado las siguientes contribuciones\\
\startitems
\printPapers{nonplusconferences}   
\stopitems
 

\subsection{EDICIÓN DE ACTAS}

El solicitante ha participado en el comité de organización de los siguientes eventos. A continuación se incluye la documentación justificativa.
\subsubsection{SPLC - 2015}. The 19th International Software Product Line Conference: New Directions in Systems and Software Product Line Engineering.
\insertpage{1}{./documentos/investigacion/revisiones/splc15.pdf}

\subsection{AÑOS COMO BECARIO FPI O SIMILAR}
El candidato ha disfrutado de la siguiente financiación durante su periodo predoctoral y postdoctoral\\
\startitems
\printFunding{all}
\stopitems

 
\subsection{OTROS MERITOS DE INVESTIGACION}
\subsubsection{Patentes}
La investigación del candidato ha estado fuertemente ligada a la transferencia y actualmente está siendo actualmente usada en la empresa OptiLabs afincada en USA como resultado del proyecto ATAACK cloud. Asimismo, los resultados de investigación del proyecto MOTIV para la detección de errores en algoritmos de seguimiento están siendo usadas por la empresa Bertin y InPixal.

Asimismo, ha patentado varias herramientas software las cuales han sido parte de los resultados de transferencia anteriormente mencionados.  Concretamente el candidato ha registrado las siguientes herramientas:\\

\printOthers{Patentes}


\subsubsection{Transferencia con empresas en USA y Francia}
La investiación del candidato esta siendo actualmente usada en la empresa OptiLabs afincada en USA como resultado del proyecto ATAACK cloud. Asímismo, los resultados de investigación del proyecto MOTIV para la detección de errores en algoritmos de seguimiento están siendo usadas por la empresa Bertin y InPixal \ref{sec:motiv}.


\subsubsection{Revisor, miembro del comité de programa y organización de conferencias}
El candidato ha formado parte del comité de programa de una diversidad de conferencias. Y ejercido como revisor para revistas indexadas en JCR. \\
\startitems
\printOthers{ComiteConf}
\stopitems


Asimismo, el candidato ha formado parte del comité de organización de las siguientes conferencias\\
\startitems
\printOthers{ComiteOrganizador} 
\stopitems

\subsubsection{Revisor revistas JCR}
\printOthers{RevisorJCR}

\subsubsection{Informes Técnicos}
\printOthers{Informe}
%--------------------------------------------------------------------------
\section{Otros méritos}
%--------------------------------------------------------------------------

\subsection{Beca de colaboración 2010-2011}
\printOthers{Beca colaboración}

\subsection{Alumno interno}
\printOthers{Alumno interno}

\subsection{Profesor asistente honorario}
\printOthers{Asistente honorario}

\subsection{Becas de postgrado}
\label{competitivo}
Contrato competitivo postdoctoral. Sección \ref{sec:cyberdefense}
\insertpage{1}{./documentos/investigacion/postdoc/postdoc_competitivo.pdf}
%\printOthers{Postdoc} 

\subsection{Conocimiento de idiomas relevantes para la investigación}
\startitems
\printOthers{Idiomas} 
\stopitems

\subsection{Premios oficiales o prestigiosos no contempladas en el apartado III}
\label{sec:premiostesis}
\printOthers{Premios} 

\subsection{Asistencia a cursos, congresos o reuniones científicas no valorados en el apartado III}
\printOthers{Otras conferencias}
 
\subsection{Actividad docente universitaria no valorable en el apartado II}
\label{sec:docenciaingles}

\subsubsection{Docencia en inglés}
\printOthers{Idiomas docencia}

\subsubsection{Cursos y seminarios impartidos orientados a la formación didáctica universitaria}
\printOthers{Cursos impartidos}

%\subsubsection{Elaboración de material docente}

 
\subsection{Otros Meritos}

\subsubsection{Acreditaciones}
\startitems
\printOthers{Acreditaciones}
\stopitems


\subsubsection{Gestión}
\startitems
\printOthers{Gestion}
\stopitems

\subsubsection{Impacto}
La investigación del candidato ha tenido un impacto relevante en el área de líneas de producto software y la configuración de productos. Algunos indicadores del mencionado impacto se presentan en esta subsección.\\
 
\startitems
\printOthers{Impacto}
\stopitems