En el plano docente, el candidato ha impartido cursos y asignaturas tanto en la Universidad de Rennes 1 dentro de sus actividades contractuales, como profesor visitante en la Universidad estatal de Milagro y recientemente en la universidad de Sevilla. Docencia, que en todos los casos está relacionada con la ingeniería del software y que ha sido impartida en español como hablante nativo, en inglés con una certificación C1 y en francés con una certificación B2 según el marco europeo.

En las actividades de docencia cabe destacar la actividad docente desarrollada en inglés y francés por parte del candidato, así como la elaboración de manuales en inglés para los alumnos extranjeros de la escuela de ingenieros INSA. Asimismo, el candidato ha desarrollado actividades de docencia en Ecuador como profesor visitante a la vez que codirige una tesis en INRIA Francia y otra en la Universidad de Sevilla en calidad de codirector. 

Las asignaturas impartidas son las siguientes:
\begin{itemize}
\item Verification et validation: Esta asignatura se cursa en el master de ingeniería informática de la universidad de Rennes 1. En ella se imparten conocimientos similares a los de la asignatura diseño y pruebas de la universidad de Sevilla. Concretamente, los conceptos comunes son las bases del diseño orientado a pruebas, el desarrollo y diseño de baterías de pruebas para sistemas de información basados en la web. No obstante, también se enseñan metodologías para pruebas de caja negra y generación de pruebas automáticas. La docencia de esta asignatura se imparte en inglés y francés.
\item Programation par contraintes: Esta asignatura se imparte en el último curso de master de la escuela de ingenieros INSA adscrita a la universidad de Rennes 1. En esta asignatura se imparten conceptos similares a los de ingeniaría de requisitos y fundamentos de programación. Concretamente, se imparten conceptos comunes de programación orientada a objetos y elicitación de requisitos. Asimismo, el resto del curso se focaliza en la programación basada en restricciones. En esta asignatura cabe destacar que el candidato ha desarrollado material en inglés para la asignatura y que ha impartido la docencia en inglés y en francés.
\item Introducción a la Ingeniería del Software y los Sistemas de Información (Universidad de Sevilla). En esta asignatura se imparten contenidos relativos a conceptos básicos de la ingeniería del software como son la gestión de proyectos y control de versiones, conceptos básicos de los sistemas de información, bases de datos, diseño en capas etc. 
\item Gestión de procesos de servicios (Universidad de Sevilla). En esta asignatura se imparten contenidos relativos al modelado y ejecución de procesos de negocio. Asimismo, en esta asignatura se imparten técnicas para implementar modelos de proceso en sistemas de gestión de procesos. 
\item Análisis y Diseño de Datos y Algoritmos: En esta asignatura se presentan al alumno de las técnicas algorítmicas básicas que le permitirán abordar el desarrollo de programas correctos y eficientes para resolver problemas no triviales. 
\end{itemize}