José Ángel ha desarrollado su actividad profesional trabajando tanto en la empresa privada como en la universidad pública en Estados unidos, Francia y España. Sus áreas investigación son las líneas de producto software y la configuración de productos. Comenzó sus estudios de grado en la Universidad de Sevilla cursando la diplomatura de Ingeniería Informática de Gestión. Posteriormente prosiguió su carrera en los estudios de licenciatura en Ingeniería Informática y master en Ingeniería y Tecnología del software. Sus primeros trabajos de investigación estuvieron enfocados al desarrollo de herramientas de gestión de la variabilidad como FaMa y BeTTy dentro del grupo ISA de la Universidad de Sevilla, las cuales son ampliamente usadas tanto en la academia como en la industria en el contexto de líneas de producto. 
\\\\
La tesis fue financiada por la Junta de Andalucía dentro del marco de becas de excelencia Talentia para el desarrollo de doctorados internacionales y por el gobierno francés mediante el concurso a diversos proyectos competitivos a nivel nacional. Nótese, que las becas Talentia el programa de postgrado más competitivo de la región andaluza equiparable en muchos aspectos a las becas FPU y FPI de la Junta de Andalucía. También cabe destacar que los puestos de doctorando con cargo a proyecto en Francia se equipararían a las becas FPI con cargo a proyectos en España. En su tesis se explora el uso del análisis automático de modelos en sistemas de alta variabilidad. Cabe destacar que la tesis se ha realizado en tres países distintos requiriendo estancias de más de un año en USA, Francia y España obteniendo el título de doctor por la Universidad de Sevilla y la Universidad de Rennes 1 en marzo de 2015 mediante un programa de cotutela financiado por la Universidad de Sevilla e INRIA. Concretamente la tesis se centra en: i) la selección y priorización de pruebas. ii) la configuración de sistemas de variabilidad distribuidos. iii) la gestión de la evolución de los sistemas de alta variabilidad. 
\\\\
Su investigación sobre líneas de producto software, sistemas de alta variabilidad y configuración de productos se ha realizado en primer lugar, dentro del marco del proyecto ATAACK cloud financiado por el gobierno de Estados unidos y desarrollado dentro de la universidad Virginia Tech, donde desarrolló un sistema cloud para el testing eficiente de aplicaciones Android. Cabe remarcar que estos resultados han inspirado la creación de una empresa en Estados Unidos. En segundo lugar, los resultados de la tesis se validaron el contexto de la optimización del testing de algoritmos de análisis de video. Esta segunda validación se enmarca en el contexto del proyecto MOTIV del gobierno francés, el cual se cierra con un total de tres empresas colaboradoras usando el software desarrollado.
\\\\
Ha desarrollado su actividad postdoctoral investigadora en el grupo de investigación Diverse en el centro de investigación INRIA. Esta actividad se desarrollaba principalmente dentro del proyecto Cybersecurity financiado por el departamento de defensa francés. Este proyecto se centró en el desarrollo guiado por modelos para la generación de políticas de filtrado para protocolos de comunicación. No obstante, José Ángel compaginó actividades de investigación en otras áreas como son el desarrollo de técnicas para el análisis de big data o la gestión de la variabilidad en sistemas altamente configurables. Estas actividades se complementan con la colaboración en la organización de distintas conferencias y workshops donde el candidato colabora como miembro del comité de programa. 
\\\\
Este grado de internacionalización ha permitido al candidato obtener diversos méritos investigadores. Concretamente, el candidato tiene un índice h de 12 según Google scholar (i10 de 12) y tiene más de 420 citas. Codirige en la actualidad un total de dos tesis doctorales en España habiendo participado en la tutorización de una en INRIA/Francia. El candidato ha publicado más de 30 trabajos de investigación dentro de los cuales nueve han sido publicados en revistas indexadas en JCR y conferencias destacadas del área. Esta visibilidad de su investigación ha habilitado al candidato para estar dentro de distintos comités de programa, la edición de actas y la colaboración en la organización de distintas conferencias y workshops líderes del área en la que investiga. El candidato mantiene contacto con investigadores de más de doce universidades distintas entre las que se encuentran universidades como Virginia Tech o Cambridge, y ha compartido autoría con más de 30 autores de reconocido prestigio.   