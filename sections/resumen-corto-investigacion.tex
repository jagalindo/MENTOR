A modo de resumen el candidato tiene los siguientes méritos en investigación:

\begin{itemize}
    \item El candidato ha realizado estancias de larga duración en centros internacionales muy prestigiosos como son INRIA en Francia y Virginia Tech en estados unidos. Asimismo, el candidato ha realizado varias estancias para tareas de seguimiento doctoral en la Universidad Estatal de Milagro en Ecuador y mantiene contacto permanente con la Universidad de los Andes en Colombia entre otras en Latinoamérica. Ha publicado trabajos conjuntos con más de 30 autores internacionales y colaborado con más de 10 instituciones distintas.  
    \begin{itemize}
        \item Estancias postdoctorales: 22 meses  
        \item Estancias pre-doctorales: 18 meses   
    \end{itemize}
\item El candidato ha publicado 9 publicaciones en revistas indexadas en el índice JCR, una revista no indexada.
\item El candidato ha publicado 7 artículos en las actas de congresos con criterios de selección equiparables a revistas JCR según la ANECA.   
\item Índice H 12 - Índice i10: 12  (Google Scholar)  
\item Citas: 420  (Google Scholar)  
\item Citas/año: Desde 2012, 70 citas    
\item Ha publicado 27 artículos en congresos del area de lineas de producto software y configuración de productos. 
\item El candidato ha obtenido un contrato postdoctoral competitivo en INRIA Francia. 
\item El candidato codirige actualmente dos tesis doctorales en la Unversidad de Sevilla. 
\item Ha participado como miembro investigador en 8 proyectos de investigación a nivel nacional francés, español y dos a nivel europeo.  
\item El candidato ha formado parte del comité organizador de la conferencia líder del área de lineas de producto software  (SPLC) en el año 2015 así en el actual 2017 y forma parte de diversos comités de programa en el ámbito nacional e internacional. 

\end{itemize}

    